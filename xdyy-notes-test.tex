\documentclass{xdyy-notes}

\xdyynotesetup{
  % 笔记信息
  info = {
    author = {夏康玮},
    title = {算子理论读书笔记},
    date = {2022年2月17日}
  },
  % 前言的落款签名
  signature = {
    name = {夏康玮},
    place = {珞珈山},
    date = {2022年2月17日}
  },
  bib = {
    % bib文件名
    resource = {xdyy-notes-template.bib},
    % bibintoc = false,
    title = { 参考文献 }
  }
}

\usepackage{zhlipsum}
\usepackage{xdyy-math}

\begin{document}

% 标题页
\maketitle

% 前言、目录部分
\frontmatter

% 前言
\begin{preface}
  \zhlipsum[1-3]
\end{preface}

% 目录
\tableofcontents

% 正文部分
\mainmatter

\chapter{第一章}

\begin{timelog}[date = {2022-02-24}]
  测试
\end{timelog}
\begin{proof}
  \begin{necessity}
    简单
  \end{necessity}

  \begin{sufficiency}
    不难
    \qedhere
  \end{sufficiency}
\end{proof}


\begin{exercise}
  这是一道题目
  \begin{enumerate}
    \item 1
    \item 2
    \item 3
  \end{enumerate}
\end{exercise}

\begin{exercise}
  这是一道题目
\end{exercise}

\section{第一节}

\begin{exercise}
  这是一道题目
\end{exercise}

\begin{exercise}
  这是一道题目
\end{exercise}

\subsection{第一节}

\begin{exercise}
  这是一道题目
\end{exercise}

\begin{exercise}
  这是一道题目
\end{exercise}


\begin{quotation}[
  book    = {《泛函分析讲义》},      % 书籍
  edition = {第二版},                % 版本
  year    = {2017},                  % 年份
  author  = {xdyy},                  % 作者
  page    = {11-14}                  % 页码
]\label{quo:test1}
  $E = m c^2$, 这 \ref{quo:test1} 是一个引用
\end{quotation}


\begin{quotation}[
  book = {Function analysis notes},
  edition = {3rd},
  year = {2022},
  author = {xdyy},
  page = {11-14}
]\label{quo:test}
  $E = m c^2$, 这 \ref{quo:test} 是一个引用
\end{quotation}


\begin{detail}[
  book = {《泛函分析讲义》},
  author = {许全华},
  edition = {第一版},
  year = {2017},
  page = {103},
  line = {倒数第三行},
  original = {
    \zhlipsum[1-2]
  }
]
  这段文字是用来测试效果的,没有实际含义
  \begin{remark}
    测试一下tcolorbox的嵌套效果
  \end{remark}
\end{detail}





\correction[
  book = {《摸鱼讲义》},
  author = {夏康玮},
  edition = {第一版},
  year = {2022},
  page = {88},
  line = {3},
  original = {
    \zhlipsum[1]
  },
  revision = {
    \zhlipsum[2]
  },
  explanation = {
    无须多言
  }
]

\section{第二节}


\zhlipsum[1-6]






\subsection{11}


\zhlipsum[1-2]



\chapter{第一章}

\section{第一节}
\cite{DiophantineQueffelec}


% 后文部分:参考文献、附录等
\backmatter
\printbibliography


\end{document}